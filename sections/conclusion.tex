Improving functional safety and quality of embedded systems is crucial due to the catastrophic consequences of their failure. Besides their applications in ragged environment, operating for a along time without interruption and the fact that such system are usually resource constrained, call for a systematic development approach. In this thesis, we have proposed several formal techniques for improving functional safety and quality of embedded systems, including requirements specifications, software system architecture, and software behavior, modelled in Simulink.

We proposed a domain-specific language for the specification of embedded systems requirements called ReSA. The language resembles the natural language English in syntax and semantics. However, its syntax is constrained in order to improve comprehensibility of the specifications. The language has formal semantics in Boolean and description logic, which has enabled for superficial and deep analysis of the requirements, e.g., consistency checking. The ReSA toolchain contains a ReSA editor that supports content-completion to guide requirements specification, and the specifications can be checked for consistency via the Z3 SMT solver. The language is validated for its expressively on a set of 300 industrial requirements, and has proven to express about 90\% of the requirements. 

We have also provided a mechanism to preserve timing and reliability requirements of multirate software applications during software allocation, while minimizing power consumption via Integer Linear Programming, ILP (exact) and Particle Optimization, PSO (meta-heuristic) methods. The ILP method is shown to be applicable to allocation of small and medium of software applications, whereas the PSO has shown to scale well for allocation of large software applications.

Furthermore, we have introduced an automated pattern-based, order-preserving method for transforming behavioral embedded systems models in Simulink into networks of stochastic timed automata analyzable by UPPPAAL SMC. The method is implemented in our tool SIMPPAAL that we have integrated with ReSA and validated on the BBW industrial prototype. 

The ReSA language is a descriptive and intuitive language as it resembles natural language. By extending its constructs to encompass design constraints, it should be possible and in fact beneficial to synthesize a high-level architecture, using correct-by-construction. Furthermore, by raising detailed hardware platform specifications in to the system design, e.g., memory, CPU, power specification, effective software to hardware allocation can be achieved. The proposed formal analysis of Simulink models can be generalized to other language that comply to a data-flow programming paradigms, e.g., LabView. 
